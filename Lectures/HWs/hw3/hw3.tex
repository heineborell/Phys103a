\documentclass[12pt]{article}
\usepackage{fullpage,enumitem,amsmath,amssymb,graphicx}

\title{BS103 Homework-3 }
%\author{Deniz O. Devecioglu  -- \texttt{dodeve@hotmail.com}}

\begin{document}
  \date{}
  \maketitle,
  \begin{center}

  \vspace{-0.3in}
  \begin{tabular}{rl}
  Due: May 9 2025 & 
  \end{tabular}
  \end{center}
  This is an individual assignment; you must solve it by yourself. (Please write down the solutions on A4 paper, not a notebook.)
  \noindent

  \rule{\linewidth}{0.4pt}
\section*{Problem 1}
A uniform board of length $L$ is sliding along a smooth, frictionless, horizontal plane as shown in Fig. \ref{fig1} a. The board then slides across the boundary with a rough horizontal surface. The coefficient of kinetic friction between the board and the second surface is $\mu_{k}$. 
\begin{enumerate}[label=(\roman*)]
\item Find the acceleration of the board at the moment its front end has travelled distance $x$ beyond the boundary.

\item The board stops at the moment its back end reaches the boundary, as shown in Fig. \ref{fig1} b. Find the initial speed of the board.
\end{enumerate}

 \section*{Problem 2}
The elastic cord of the catapult shown in Fig. \ref{fig2} has a total relaxed length of $2l_{0}$; its ends are attached to fixed supports at a distance $2b$ apart.
\begin{enumerate}[label=(\roman*)]
\item An object of mass $m$ is placed at the midpoint of the cord and drawn back to point B. Show that when it is released, the object travels along the line $BC$, and at the point $C$, it begins its free flight with speed $v=\sqrt{2k/m}(l-l_{0})$. Use conservation of energy in this case.
\item Show that if the tension developed in the cord is proportional to the increase in its length, the component of force in the $x$ direction is
\begin{align}
F_{x}=2kb\left(\frac{1}{\sin\theta}-\frac{1}{\sin\theta_{0}}\right)\cos\theta.
\end{align}
\item Noting that the position of the stone in the catapult is given by $x=-b\cot\theta$, derive the expression for the work done in a moving distance $dx$. Then integrate this expression between $\theta_{0}$ and $\theta$ to find the total work done in extending the catapult. Finally, from work done, find the speed of the stone leaving the catapult and show it it's equal to the speed $v$ you found in part i).
\end{enumerate}




\section*{Problem 3}
Suppose that the potential energy for a particle moving along the $x$ axis is 
\begin{align}
U(x)=\frac{b}{x^{2}}-\frac{2c}{x}
\end{align}
where $b$ and $c$ are positive constants
\begin{enumerate}[label=(\roman*)]

\item Plot $U(x)$ as a function of $x$: assume $b=c=1$ for this purpose. Where is the equilibrium point?

\item Suppose the energy of the particle is $E=-\frac{1}{2}c^{2}/b$. Find the turning points of the motion.

\item Suppose the energy of the particle is $E=\frac{1}{2}c^{2}/b$. Find the turning points of the motion. How many turning points are there in this case?

\end{enumerate}
\section*{Problem 4}
A nucleus A of mass $2m$, travelling with a velocity $u$, collides with a stationary nucleus of mass $10m$. The collision results in a change in the total kinetic energy. After the collision, the nucleus $A$ is observed to be travelling with speed $v_{1}$ at $90^{\circ}$ to its original direction of motion, and B is travelling with a speed $v_{2}$ at an angle $\theta$ ($\sin\theta=3/5$) to the original direction of A
\begin{enumerate}[label=(\roman*)]

\item What are the magnitudes of $v_{1}$ and $v_{2}$?

\item What fraction of the initial kinetic energy is gained or lost as a result of the interaction?

\end{enumerate}
\section*{Problem 5}
A thin, uniform rod is bent in the shape of a semicircle of radius $R$ (see Fig \ref{fig3}). Where is the center of mass of this rod?

\section*{Problem 6}
A boat of mass $M$ and length $L$ is floating in the water, stationary: a man of mass $m$ is sitting at the bow. The man stands up, walks up to the stern of the boat, and sits down again. 
\begin{enumerate}[label=(\roman*)]

\item If the water is assumed to offer no resistance at all to the motion of the boat, how far does the boat move as a result of the man's trip from bow to stern? (Use the fact that the center of mass of the system (man+boat) does not change, since there is no external force.)

\item More realistically, assume that the water offers a viscous resistance given by $-kv$, where $k$ is a constant and $v$ is the velocity of the boat. Show that in this case, one has the remarkable result that the boat should eventually return to its initial position.

\item (Optional for those interested) Consider the paradox presented by the fact that, according to (ii), any nonzero value of $k$, however small, implies that the boat ends up at its starting point, but a strictly zero value implies that it ends up somewhere else. How do you explain this discontinuous jump in the final position when the variation of $k$ can be imagined as continuous, down to zero? For enlightenment, see the short and clear analysis by D. Tilley. \textit{Am. J. Phys.}, 35, 546 (1967).

\end{enumerate}
\begin{figure}
\center
\includegraphics[scale=0.5]{setup1}
\caption{Figure for problem 1}\label{fig1}
\end{figure}

\begin{figure}
\center
\includegraphics[scale=0.5]{setup4}
\caption{Figure for problem 2}\label{fig2}
\end{figure}

\begin{figure}
\center
\includegraphics[scale=0.8]{setup3}
\caption{Figure for problem 5}\label{fig3}
\end{figure}

\end{document}






