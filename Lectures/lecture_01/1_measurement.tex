% Created 2026-02-12 목 19:37
%& /home/deniz/.cache/org-persist/95/033ed6-430d-4089-afec-ec62397eb371-80dc43518c2fdc2b1c61d973b63eb550
% Created 2026-02-12 목 19:37
% Intended LaTeX compiler: pdflatex
\documentclass[presentation]{beamer}
\usepackage[utf8]{inputenc}
\usepackage[T1]{fontenc}
\usepackage{amsmath}
\usepackage{amssymb}
\usepackage{capt-of}
\usepackage{hyperref}
\usepackage{booktabs}
\usepackage{amsmath}

%% ox-latex features:
%   !announce-start, !guess-pollyglossia, !guess-babel, !guess-inputenc,
%   maths, image, !announce-end.

\usepackage{amsmath}
\usepackage{amssymb}

\usepackage{graphicx}

%% end ox-latex features


% end precompiled preamble
\ifcsname endofdump\endcsname\endofdump\fi

\usetheme{default}
\author{Phys105}
\date{Jan 16, 2025}
\title{CHP-1 Measurement}
\hypersetup{
 pdfauthor={Phys105},
 pdftitle={CHP-1 Measurement},
 pdfkeywords={},
 pdfsubject={},
 pdfcreator={},
 pdflang={English}}
\begin{document}

\maketitle
\begin{frame}{Outline}
\tableofcontents
\end{frame}

\section*{Objectives}
\label{sec:org921f3b5}

\begin{itemize}
\item What does physics do?

\item Understand units.

\item What are the fundamental units.

\item Conversions.

\item Significant Figures.
\end{itemize}
\section*{Physics and Experiment}
\label{sec:org55db410}
\begin{itemize}
\item Like all other sciences, physics is based on experimental observations
\end{itemize}
and quantitative measurements.

\begin{itemize}
\item The main objectives of physics are to identify a limited number of fundamental laws that govern natural phenomena and use them to develop theories that can predict the future experiments.

\item When there is a discrepancy between the prediction of a theory and experimental results, new or modified theories must be formulated.

\item For example:
\begin{itemize}
\item Newtonian mechanics \(\rightarrow\) Einstein theory of relativity
\item Blackbody radiation \(\rightarrow\) Quantum mechanics
\end{itemize}
\end{itemize}
\section*{Standards of Length, Mass, Time}
\label{sec:orgc75005d}
\begin{itemize}
\item To describe natural phenomena we need measurements.
\end{itemize}
\#+begin\textsubscript{quote}
\alert{Base quantities}: Fortunately, they are not all independent; for example, speed is the ratio of a length to a time.
\#+end\textsubscript{quot}
\begin{itemize}
\item In mechanics we have three fundamental quantities: \emph{length, mass} and \emph{time}.
\item In 1960 a committee established set of standards (Kelvin, Ampere, Candela, Mole).
\end{itemize}
\section*{Oilfield Units (The "Cursed" System)}
\label{sec:orgf67f6bb}

\begin{itemize}
\item A very good video on how chaotic the units in oil-field engineering are:
\item \href{https://www.youtube.com/watch?v=sdWEGzWFcCc}{Oilfield Units: A Measurement System so Cursed it made me Change Career}
\end{itemize}
\begin{center}
\includegraphics[width=.9\linewidth]{oilfield_units.png}
\end{center}
\section*{Length}
\label{sec:org53f4a04}
\begin{itemize}
\item \alert{meter (1960)}: Distance between two lines on a specific platinum-iridium bar in France.
\end{itemize}
\begin{center}
\includegraphics[width=.9\linewidth]{ss_3.jpg}
\end{center}
\begin{itemize}
\item \alert{meter (1983)}: Distance traveled by light in vacuum during a time of \(1/299,792,458\) second.
\end{itemize}
\section*{Time}
\label{sec:org119fc26}
\begin{itemize}
\item In 1967 second is defined as \(9,162,631,770\) times the period of vibration of radiation from the cesium-133 atom.
\end{itemize}
\begin{center}
\includegraphics[width=.9\linewidth]{output-3.jpg}
\end{center}
\begin{itemize}
\item A cesium fountain atomic clock. The clock will neither gain nor lose a second in 20 million years.
\end{itemize}
\section*{Mass}
\label{sec:orga24b09f}
\begin{itemize}
\item The SI standard of mass is a cylinder of platinum and iridium kept at the International Bureau of Weights and Measures.
\end{itemize}
\begin{center}
\includegraphics[width=.9\linewidth]{output-0.jpg}
\end{center} \begin{center}
\includegraphics[width=.9\linewidth]{output-4.jpg}
\end{center}
\section*{Kibble Balance}
\label{sec:org7d74438}
\begin{itemize}
\item In a Kibble balance, a standard mass can be measured when the downward pull of gravity is balanced by magnetic force.
\end{itemize}
\begin{center}
\includegraphics[width=.9\linewidth]{output-5.jpg}
\end{center}
\section*{Derived Units \& Dimensional Analysis}
\label{sec:org37a0e98}
\section*{Dimensional Analysis}
\label{sec:org1cb61b2}
\begin{itemize}
\item We use brackets \([ ]\) to denote dimensions.
\item \(\text{speed} \rightarrow [v] = L/T\)
\item \(\text{area} \rightarrow [A] = L^2\)

\item A dimensional analysis example:
\end{itemize}
\begin{center}
\includegraphics[width=.9\linewidth]{output-6.jpg}
\end{center}
\begin{frame}[label={sec:org8befdc9}]{The G.I. Taylor Bomb Example}
\begin{itemize}
\item Taylor assumed: \(r\) (radius), \(\rho\) (density of air), \(E\) (energy), \(t\) (time).
\item The radius is then:
\[ r(t) = C \rho^{-1/5} E^{1/5} t^{2/5} \]
\end{itemize}
\end{frame}
\section*{Conversion}
\label{sec:orgb3f30c0}
\section*{Conversion}
\label{sec:orgfd9a0f4}
\begin{itemize}
\item We use \alert{chain-link} conversion.
\end{itemize}
\begin{center}
\includegraphics[width=.9\linewidth]{output-1.jpg}
\end{center}

\begin{itemize}
\item \alert{Interesting Case}: Fuel efficiency (Litres/100km).
\begin{itemize}
\item Calculated as: 
\[ \frac{1000 \text{ cm}^3}{100 \text{ km}} = 0.01 \text{ mm}^2 \]
\end{itemize}
\end{itemize}
\begin{center}
\includegraphics[width=.9\linewidth]{droppings_car.png}
\end{center}

\begin{center}
\includegraphics[width=.9\linewidth]{birds.png}
\end{center}
\section*{Significant Figures}
\label{sec:orgaeb3d5d}
\begin{frame}[label={sec:org17d847f}]{Rules for Significant Figures}
\begin{itemize}
\item Non-zero digits are significant (\(365 \rightarrow 3\)).
\item Zeros between digits are significant (\(6.003 \rightarrow 4\)).
\item Leading zeros are \alert{not} significant (\(0.0075 \rightarrow 2\)).
\item Trailing decimal zeros are significant (\(3.50 \rightarrow 3\)).
\end{itemize}
\end{frame}
\begin{frame}[label={sec:org7868ca9}]{Calculation Rules}
\begin{itemize}
\item \alert{Multiplication/Division}: Smallest number of significant figures.
\[ A = \pi r^2 = \pi (6.0 \text{ cm})^2 = 1.1 \times 10^2 \text{ cm}^2 \]
\item \alert{Addition/Subtraction}: Smallest number of decimal places.
\begin{itemize}
\item Example: \(23.2 + 5.174 = 28.4\)
\end{itemize}
\end{itemize}
\end{frame}
\end{document}
