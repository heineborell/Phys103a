% !TeX TS-program = lualatex
\documentclass[aspectratio=169]{beamer}

\usepackage{fontspec}
\usepackage{unicode-math}

% Text font
\setsansfont{Iosevka}

% Math font (Fira)
\setmathfont{Fira Math}

% Optional: make math match sans-serif look better
\unimathsetup{
  math-style = ISO,
  bold-style = ISO
}

\title{CHP-1 Measurement}
\author{D.O.D}
\date{\today}

\begin{document}

\maketitle

\begin{frame}{Table of contents}
  \setbeamertemplate{section in toc}[sections numbered]
  \tableofcontents[hideallsubsections]
\end{frame}
\section{Objectives}

\begin{frame}[fragile]{Objectives}
  \begin{itemize}
    \item What does physics do?
    \item Understand units.
    \item What are the fundamental units.
    \item Conversions.
    \item Significant Figures.
  \end{itemize}

\end{frame}

\section{Physics and Experiment}

\begin{frame}{Physics and Experiment}
  \begin{itemize}
    \item Like all other sciences, physics is based on experimental
      observations and quantitative measurements.

    \item The main objectives of physics are to identify a limited
      number of fundamental laws that govern natural phenomena and
      use them to develop theories that can predict the future
      experiments. (Search for Thomas Kuhn, Karl Popper).

    \item When there is a discrepancy between the prediction of a
      theory and experimental results, new or modified theories
      must be formulated to remove discrepancy.

    \item For example (Newtonian mechanics → Einstein theory of
      relativity) (Blackbody radiation → Quantum mechanics )
  \end{itemize}
\end{frame}
\begin{frame}
  \begin{align}
    \oint \vec{B} \cdot d\vec{l}& = \mu_0\\
    \vec F &=m \vec a
  \end{align}
\end{frame}

\section{Standarts of Length, Mass, Time}
\begin{frame}{Standarts of Length, Mass, Time}
  - To describe natural phenomena we need measurements.\\

  \begin{alertblock}{Base quantities}

    There are so many physical quantities that it is a problem to
    organize them. Fortunately, they are not all independent; for
    example, speed is the ratio of a length to a time.

  \end{alertblock}
  - In mechanics we have three fundamental quantities length,
  mass and time.\\
  - In order to reproduce experiments we need standards.
\end{frame}
\end{document}
