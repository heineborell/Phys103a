\documentclass[10pt]{beamer}

% Theme selection
\usetheme{metropolis}
\usepackage{appendixnumberbeamer}
\usepackage{booktabs}
\usepackage[scale=2]{ccicons}
\usepackage{pgfplots}
\usepgfplotslibrary{dateplot}
\usepackage{geometry}
\usepackage{amsmath}
\usepackage{tikz}
\usetikzlibrary{shapes.geometric, arrows, positioner}

% Font settings (requires XeLaTeX or LuaLaTeX to match your fonts)
% If using pdfLaTeX, these lines should be commented out.
\usepackage{fontspec}
\setmonofont{Iosevka}
% Note: "Adwaita Mono" and "Noto Sans Math" must be installed on your OS

\title{CHP-1 Measurement}
\subtitle{Phys105}
\date{January 16, 2025}
\author{}
\institute{}

\begin{document}

\maketitle

\begin{frame}{Table of Content}
  \setbeamertemplate{section in toc}[sections numbered]
  \tableofcontents[hideallsubsections]
\end{frame}

\begin{frame}{Objectives}
  \begin{itemize}
    \item What does physics do?
    \item Understand units.
    \item What are the fundamental units.
    \item Conversions.
    \item Significant Figures.
  \end{itemize}
\end{frame}

\section{Physics and Experiment}

\begin{frame}{Physics and Experiment}
  \begin{itemize}
    \item Like all other sciences, physics is based on experimental
      observations and quantitative measurements.
    \item The main objectives of physics are to identify a limited
      number of fundamental laws that govern natural phenomena and
      use them to develop theories that can predict the future
      experiments. (Check out for Thomas Kuhn, Karl Popper).
    \item When there is a discrepancy between the prediction of a
      theory and experimental results, new or modified theories must
      be formulated to remove discrepancy.
    \item For example:
      \begin{itemize}
        \item Newtonian mechanics $\rightarrow$ Einstein theory of relativity
        \item Blackbody radiation $\rightarrow$ Quantum mechanics
      \end{itemize}
  \end{itemize}
\end{frame}

\section{Standards of Length, Mass, Time}

\begin{frame}{Standards of Length, Mass, Time}
  \begin{itemize}
    \item To describe natural phenomena we need measurements.
  \end{itemize}

  \begin{center}
    \fbox{
      \begin{minipage}{0.8\textwidth}
        \textbf{Base quantities}: There are so many physical
        quantities that it is a problem to organize them.
        Fortunately, they are not all independent; for example, speed
        is the ratio of a length to a time.
      \end{minipage}
    }
  \end{center}

  \begin{itemize}
    \item In mechanics we have three fundamental quantities
      \textit{length, mass} and \textit{time}.
    \item In order to reproduce experiments we need \textit{standards}.
    \item In 1960 a committee established a set of standards
      (including Kelvin, Ampere, Candela, Mole).
  \end{itemize}
\end{frame}

\begin{frame}{Unit Chaos}
  \begin{itemize}
    \item A very good video on how chaotic units in oil-field engineering are:
  \end{itemize}
  \begin{figure}
    %\includegraphics[width=0.8\textwidth]{oilfield_units.png}
    \url{https://www.youtube.com/watch?v=sdWEGzWFcCc}
  \end{figure}
\end{frame}

\begin{frame}{Unit Systems}
  \begin{columns}
    \column{0.5\textwidth}
    \centering
    \small
    \begin{tikzpicture}[node distance=1.2cm, every node/.style={draw,
      fill=blue!10, font=\scriptsize}]
      \node (A) {SI Units};
      \node (B) [below of=A, xshift=-0.8cm] {Meter};
      \node (C) [below of=A] {Kilogram};
      \node (D) [below of=A, xshift=0.8cm] {Second};
      \draw [->] (A) -- (B); \draw [->] (A) -- (C); \draw [->] (A) -- (D);
    \end{tikzpicture}

    \column{0.5\textwidth}
    \centering
    \small
    \begin{tikzpicture}[node distance=1.2cm, every node/.style={draw,
      fill=red!10, font=\scriptsize}]
      \node (A) {US Customary};
      \node (B) [below of=A, xshift=-1cm, yshift=0.3cm] {Inch/Foot};
      \node (C) [below of=A, xshift=-0.3cm, yshift=-0.2cm] {Ounce/Pound};
      \node (D) [below of=A, xshift=0.3cm, yshift=-0.2cm] {Second};
      \node (E) [below of=A, xshift=1cm, yshift=0.3cm] {Gallon};
      \draw [->] (A) -- (B); \draw [->] (A) -- (C); \draw [->] (A) --
      (D); \draw [->] (A) -- (E);
    \end{tikzpicture}
  \end{columns}
\end{frame}

\begin{frame}{Length}
  \begin{itemize}
    \item Distance between two points in space.
    \item \textit{Yard}: Tip of nose to end of arm of the King of England!
    \item Until 1799: 1/10,000,000 of distance from equator to North
      Pole via Paris.
    \item 1960: Distance between lines on a platinum-iridium bar.
  \end{itemize}
  \begin{figure}
    %\includegraphics[width=0.5\textwidth]{ss_3.jpg}
    \caption{Platinum-iridium bar.}
  \end{figure}
  \begin{itemize}
    \item 1983: Distance traveled by light in vacuum in 1/299,792,458 second.
  \end{itemize}
\end{frame}

\begin{frame}{Time}
  \begin{itemize}
    \item Pre-1967: Mean solar day. Second = $(1/60) \times (1/60)
      \times (1/24)$ of a solar day.
    \item 1967: 9,162,631,770 periods of radiation from cesium-133 atom.
  \end{itemize}
  \begin{columns}
    \column{0.4\textwidth}
    %\includegraphics[width=\textwidth]{output-3.jpg}
    \column{0.6\textwidth}
    A cesium fountain atomic clock. Will not lose a second in 20 million years.
  \end{columns}
\end{frame}

\begin{frame}{Mass}
  \begin{itemize}
    \item Standard: Cylinder of platinum-iridium in France (1 kg).
    \item Kibble Balance: Measuring mass via magnetic fields and
      electrical currents linked to quantum constants.
  \end{itemize}
  %\includegraphics[width=0.6\textwidth]{output-5.jpg}
\end{frame}

\begin{frame}{Dimensional Analysis}
  \begin{itemize}
    \item Symbols: $L$ (Length), $M$ (Mass), $T$ (Time).
    \item Brackets $[ \ ]$ denote dimensions.
    \item Speed $\rightarrow [v] = L/T$
    \item Area $\rightarrow [A] = L^2$
  \end{itemize}
\end{frame}

\begin{frame}{The Bomb Example (G.I. Taylor)}
  Estimate of atomic bomb spreading rate (1940s):
  \[ r(t) = C \rho^{-1/5} E^{1/5} t^{2/5} \]
  Where $r$ is radius, $\rho$ is air density, $E$ is energy, and $t$ is time.
\end{frame}

\section{Conversion}

\begin{frame}{Conversion}
  \begin{itemize}
    \item \textbf{Chain-link conversion}: Multiplying by a ratio equal to unity.
    \item Example: Fuel efficiency.
      \[ \text{litres/100km} = \frac{10^{-3} m^3}{10^5 m} = 0.01 \text{ mm}^2 \]
    \item \textit{Interpretation}: The cross-sectional area of a tube
      of gas stretched along your route.
  \end{itemize}
\end{frame}

\begin{frame}{Famous Errors}
  \begin{itemize}
    \item \textbf{Mars Climate Orbiter (1999)}: \$125 million lost
      because Lockheed Martin used English units (pounds) while NASA
      used Metric (Newtons).
    \item \textbf{Tokyo Disneyland (2004)}: Space Mountain derailment
      due to an axle sized in inches instead of millimeters.
  \end{itemize}
\end{frame}

\section{Significant Figures}

\begin{frame}{Significant Figures: Rules}
  \begin{itemize}
    \item Non-zero digits are significant.
    \item Zeros between digits are significant (6.003 $\rightarrow$ 4).
    \item Leading zeros are NOT significant (0.0075 $\rightarrow$ 2).
    \item Trailing decimal zeros are significant (3.50 $\rightarrow$ 3).
  \end{itemize}
\end{frame}

\begin{frame}{Calculations}
  \begin{itemize}
    \item \textbf{Mult/Div}: Result has same sig figs as the factor
      with the \textit{fewest} sig figs.
    \item \textbf{Add/Sub}: Result has same decimal places as the
      term with the \textit{fewest} decimal places.
  \end{itemize}
\end{frame}

\end{document}
