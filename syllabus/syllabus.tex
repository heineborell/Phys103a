\documentclass[12pt, a4paper]{article}
%\usepackage{geometry}
\usepackage[inner=2.0cm,outer=2.0cm,top=2.5cm,bottom=2.5cm]{geometry}
\pagestyle{empty}
\usepackage{graphicx}
\usepackage{CJKutf8}
\usepackage{fancyhdr, lastpage, bbding, pmboxdraw}
\usepackage[usenames,dvipsnames]{color}
\definecolor{darkblue}{rgb}{0,0,.6}
\definecolor{darkred}{rgb}{.7,0,0}
\definecolor{darkgreen}{rgb}{0,.6,0}
\definecolor{red}{rgb}{.98,0,0}
\usepackage[colorlinks,pagebackref,pdfusetitle,urlcolor=darkblue,citecolor=darkblue,linkcolor=darkred,bookmarksnumbered,plainpages=false]{hyperref}
\renewcommand{\thefootnote}{\fnsymbol{footnote}}
\newcommand{\RNum}[1]{\uppercase\expandafter{\romannumeral #1\relax}}
\pagestyle{fancyplain}
\fancyhf{}
\lhead{ \fancyplain{}{\textsc{General Physics II} }}
\chead{ \fancyplain{}{} }
\rhead{ \fancyplain{}{\textsc{Bs105a} }}
%\rfoot{\fancyplain{}{page \thepage\ of \pageref{LastPage}}}
\fancyfoot[RO, LE] {page \thepage\ of \pageref{LastPage} }
\thispagestyle{plain}

%%%%%%%%%%%% LISTING %%%
\usepackage{listings}
\usepackage{caption}
\DeclareCaptionFont{white}{\color{white}}
\DeclareCaptionFormat{listing}{\colorbox{gray}{\parbox{\textwidth}{#1#2#3}}}
\captionsetup[lstlisting]{format=listing,labelfont=white,textfont=white}
\usepackage{verbatim} % used to display code
\usepackage{fancyvrb}
\usepackage{acronym}
\usepackage{amsthm}
\VerbatimFootnotes % Required, otherwise verbatim does not work in footnotes!
\definecolor{OliveGreen}{cmyk}{0.64,0,0.95,0.40}
\definecolor{CadetBlue}{cmyk}{0.62,0.57,0.23,0}
\definecolor{lightlightgray}{gray}{0.93}
\lstset{
  %language=bash,                          % Code langugage
  basicstyle=\ttfamily,                   % Code font, Examples:
  % \footnotesize, \ttfamily
  keywordstyle=\color{OliveGreen},        % Keywords font ('*' = uppercase)
  commentstyle=\color{gray},              % Comments font
  numbers=left,                           % Line nums position
  numberstyle=\tiny,                      % Line-numbers fonts
  stepnumber=1,                           % Step between two line-numbers
  numbersep=5pt,                          % How far are line-numbers from code
  backgroundcolor=\color{lightlightgray}, % Choose background color
  frame=none,                             % A frame around the code
  tabsize=2,                              % Default tab size
  captionpos=t,                           % Caption-position = bottom
  breaklines=true,                        % Automatic line breaking?
  breakatwhitespace=false,                % Automatic breaks only at whitespace?
  showspaces=false,                       % Dont make spaces visible
  showtabs=false,                         % Dont make tabls visible
  columns=flexible,                       % Column format
  morekeywords={__global__, __device__},  % CUDA specific keywords
}

%%%%%%%%%%%%%%%%%%%%%%%%%%%%%%%%%%%%
\begin{document}
\begin{center}
  {\Large \textsc{Bs103a (General Physics I)}}
\end{center}
\begin{center}
  2026 Spring\\
  DGIST \\
\end{center}
%\date{September 26, 2014}

\begin{center}
  \rule{\textwidth}{0.4pt}
  \begin{minipage}[t]{\textwidth}
    \medskip
    \begin{tabular}{lll}
      \textbf{Instructor:} & Deniz Olgu Devecio\u{g}lu &  \medskip \\
      \textbf{Office:} & L06 & \medskip \\
      \textbf{Contact:} & deniz@dgist.ac.kr & \medskip\\
      \textbf{Office Hours:} & TBA & \medskip\\

    \end{tabular}\medskip
  \end{minipage}
  \rule{\textwidth}{0.4pt}
\end{center}
\vspace{.2cm}
\setlength{\unitlength}{1in}
\renewcommand{\arraystretch}{2}

\noindent\textbf{Description:}
This calculus based course provides students with the basic concepts
of physics that enable them to understand describe and explain
natural phenomena. Emphasis is laid on general principles and
fundamental concepts in mechanics with applications of physics in
various fields of engineering.

\vskip.15in
\noindent\textbf{Course times:}

\begin{enumerate}
    All lectures are at E7-241.
  \item  General Physics II - Section 1 Tuesday 15:00-16:30 /
    Thursday 15:00-16:30.
  \item  General Physics II - Section 2 Tuesday 16:30-18:00 /
    Thursday 16:30-18:00.
  \item  General Physics II- Section 3 Monday 10:30-12:00 /
    Wednesday 10:30-12:00.
\end{enumerate}

\vskip.15in

\noindent\textbf{Course Pages:}
\begin{enumerate}
  \item LMS: All course content such as announcements, slides,
    homework, grades and other required reading will be made
    available on LMS.
\end{enumerate}

\vskip.15in
\noindent\textbf{Teaching Assistants:}

\begin{CJK}{UTF8}{mj} % "mj" = Korean font (MyoungJo)
  \flushleft

  \begin{CJK}{UTF8}{mj} % "mj" = Korean font (MyoungJo)
    Section 1: 이태균, email: dlxorbs4150@dgist.ac.kr\\
    Section 2: 이승우, email: leeseungwoo0313@dgist.ac.kr\\
    Section 3: 조민서, email: log0114@dgist.ac.kr
  \end{CJK}

\end{enumerate}

\vskip.15in
\noindent\textbf{Teaching Materials:}

The first three books will be used as main source depending on topic.
Since the physics syllabus is highly standardised there is basically
no difference in terms of topics covered. Check each of them and follow
the one you like.

\begin{enumerate}
\item Principles of Physics, Halliday \& Resnick, Eleventh Edition,
  QC21.3 H35 2020.
\item Physics for Scientists and Engineers,  Hans C. Ohanian, John
  T. Markert, 2007.
\item Physics for Scientists and Engineers with Modern Physics,
  Raymond Serway & John Jewett, 2013.
\end{enumerate}

\vskip.35in

\noindent \textbf{Tentative Course Schedule:}
\begin{center}
\begin{flushleft}
  The schedule below is tentative. Any unexpected changes to the
  schedule will be announced in-class (as the course proceeds).

  \dotfill Week \#1  \\
  {\color{darkgreen}{\Rectangle}} Mathematical Review; Vector
  Algebra; Gradient; Divergence;
  Line, Surface, Volume Integrals. \smallskip \\

  \dotfill Week \#2  \\
  {\color{darkgreen}{\Rectangle}} Static Electricity; Electric
  Charge and its Conservation;   Insulators and Conductors;
  Induced Charge; Coulomb’s Law; The Electric Field. \smallskip \\

  \dotfill Week \#3  \\
  {\color{darkgreen}{\Rectangle}} Electric Field of a Continuous
  Charge Distribution; Electric Field Lines; Electric Fields and
  conductors; Motion of a Charged Particle in an Electric Field;
  Electric Dipoles. \smallskip \\

  \dotfill Week \#4  \\
  {\color{darkgreen}{\Rectangle}} Electric Potential Energy and
  Potential Difference; Relation Between Electric Potential and
  Electric Field; Electric Potential due to Point Charges;
  Potential Due to Any Charge Distribution; Equipotential Lines
  and Surfaces; Applications of Electrostatics.\smallskip \\

  \dotfill Week \#5  \\
  {\color{darkgreen}{\Rectangle}} Definition of
  Capacitance;Calculation of Capacitance; Capacitors in Series
  and Parallel; Storage of Electrical Energy; Dielectrics; An
  Atomic Description of  Dielectrics.\smallskip \\

  \dotfill Week \#6  \\
  {\color{darkgreen}{\Rectangle}} Electric Current; Ohm’s Law:
  Resistance and Resistors; Resistivity; Model for Electrical
  Conduction; Resistance and  Temperature; Superconductors;
  Electrical Power.
  \smallskip \\

  \dotfill Week \#7 \\
  {\color{darkgreen}{\Rectangle}} Midterms.
  \smallskip \\

  \dotfill Week \#8  \\
  {\color{darkgreen}{\Rectangle}} EMF and Terminal Voltage;
  Resistors in Series and Parallel; Kirchhoff’s Rules; EMFs in
  Series and Parallel; Charging a Battery; RC Circuits – Resistor
  and Capacitor in Series.
  \smallskip \\

  \dotfill Week \#9  \\
  {\color{darkgreen}{\Rectangle}} Magnets and Magnetic Fields;
  Electric Current Produce Magnetic Fields; Force on an Electric
  Current in a Magnetic Field: Definition of B; Force on an
  Electric Charge Moving in a Magnetic Field; Torque on a Current Loop.
  \smallskip \\

  \dotfill Week \#10  \\
  {\color{darkgreen}{\Rectangle}} Magnetic Dipole Moment;
  Applications: Motors, Loudspeakers, Galvanometers; The Hall Effect.
  \smallskip \\

  \dotfill Week \#11  \\
  {\color{darkgreen}{\Rectangle}} The Biot-Savart Law; The
  Magnetic Force Between Two Parallel
  Conductors; Ampere's Law; The Magnetic Field of a Solenoid;
  Gauss's Law in Magnetism; Magnetism in Matter.
  \smallskip \\

  \dotfill Week \#12 \\
  {\color{darkgreen}{\Rectangle}} Induced EMF; Faraday’s Law of
  Induction; Lenz’s Law; EMF Induced in a Moving Conductor;
  Electric Generators; Back EMF and Counter Torque; Eddy
  Currents; Transformers and Transmission of Power; A Changing
  Magnetic Field Produces an electric Field.
  \smallskip \\

  \dotfill Week \#13 \\
  {\color{darkgreen}{\Rectangle}} Mutual Inductance;
  Self-Inductance; Inductors; Energy Stored in a Magnetic Field;
  LR Circuits; LC Circuits and Electromagnetic Oscillations; LC
  Oscillations with Resistance (RLC Circuit).
  \smallskip \\

  \dotfill Week \#14 \\
  {\color{darkgreen}{\Rectangle}} AC Circuits and Reactance; RLC
  Series AC Circuit; Phasor Diagrams; Resonance in AC Circuits.

  \dotfill Week \#15  \\
  {\color{darkgreen}{\Rectangle}} Changing Electric Fields
  Produce Magnetic Fields; Displacement Current; Gauss’s Law for
  Magnetism; Maxwell’s Equations; Production of Electromagnetic
  Waves; Electromagnetic Waves, and Their Speed, Derived From
  Maxwell’s Equations.
  \smallskip \\
  \dotfill Week \#16 \\
  {\color{darkgreen}{\Rectangle}} Finals.
  \smallskip \\

\end{flushleft}
\end{center}

\vspace*{.15in}
\noindent\textbf{Grading Policy:}
Three components determine your grade: Exams, Homework and
Attendance. The relative weights are as follows: \medskip\\
\begin{center}
\begin{minipage}{3.5in}
  \begin{flushleft}
    Exams \dotfill ~ $70$\% \\
    {\color{darkred}{\Rectangle}} Two exams. Midterm is $30$\%,
    Final is $40$\%.

    \medskip \\
    Homeworks \dotfill ~ $10$\% \\
    {\color{darkred}{\Rectangle}} 2 homeworks, each accounting
    for $5$\%.   \medskip \\

    {\color{darkred}{\Rectangle}} Project \dotfill ~ $25$\% \\

    {\color{darkred}{\Rectangle}} Attendance \dotfill ~ $5$\% \\

  \end{flushleft}
\end{minipage}
\end{center}
\underline{Letter range will be as follows.}

\scalebox{0.8}{
\begin{tabular}{llllllllllllll}
  \hline
  Letter & F    & D-    & D     & D+    & C-    & C     & C+    &
  B-    & B     & B+    & A-    & A     & A+     \\ \hline
  Range  & 0-20 & 21-26 & 27-33 & 34-40 & 41-46 & 47-53 & 54-60 &
  61-65 & 66-70 & 71-75 & 76-85 & 86-95 & 96-100 \\ \hline
\end{tabular}}

\vskip.15in
\noindent\textbf{Course Objectives:}
The goal of this course is to provide a calculus-based physics course
to help students pursuing advanced studies in sciences and
engineering develop conceptual understanding of physical principles,
the ability to reason, and gain skills for problem solving.

\vskip.15in
\noindent\textbf{Academic Integrity:}
You are expected to demonstrate academic honesty in all aspects of
this course. Academic dishonesty includes, but is not limited to:
violating clearly stated rules for taking an exam or completing an
assignment; plagiarism (including material from sources without a
citation and quotation marks around any borrowed words); claiming
another’s work or a modification of another’s work as one’s own;
buying or attempting to buy papers or projects for a course;
fabricating information or citations.
%%%%%% THE END
\end{document}
